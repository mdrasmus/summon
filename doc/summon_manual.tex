\documentclass[12pt]{article}
\usepackage{amsmath}
\usepackage{graphicx}
\usepackage{pslatex}  % nice fonts

% page dimensions
\textwidth=6.5in
\oddsidemargin=-0.0in
\evensidemargin=-0.0in
\topmargin=-0.5in
\footskip=0.8in
\textheight=8.50in


% list shortcuts
\newcommand{\enum}[1]{\begin{enumerate} #1 \end{enumerate}}
\newcommand{\items}[1]{\begin{itemize} #1 \end{itemize}}

% formatting macros
\newcommand{\bold}[1]{{\bf #1}}
\newcommand{\ul}[1]{\underline{#1}}
\newcommand{\code}[1]{{\tt #1}}
\newcommand{\codeblock}[1]{\vspace{.1in} {\tt #1} \vspace{.1in}}

% reference macros
\newcommand{\figref}[1]{Figure~\ref{#1}}
\newcommand{\secref}[1]{Section~\ref{#1}}
\newcommand{\algref}[1]{Algorithm~\ref{#1}}



\newcommand{\version}{1.8}


%%%%%%%%%%%%%%%%%%%%%%%%%%%%%%%%%%%%%%%%%%%%%%%%%%%%%%%%%%%%%%%%%%%%%%%%%%%%

\begin{document}

\begin{titlepage}

\begin{center}

\vspace*{2.5in}

{\huge \bf \fontfamily{phv}\selectfont 
SUMMON \version\ Manual
}
\vspace*{.5in}

{\large
Matt Rasmussen

\today
}
\vspace*{.5in}

Computer Science and Artificial Intelligence Lab

Massachusetts Institute of Technology

\vspace*{.25in}

rasmus@mit.edu
\end{center}

\end{titlepage}


\tableofcontents

\clearpage

%%%%%%%%%%%%%%%%%%%%%%%%%%%%%%%%%%%%%%%%%%%%%%%%%%%%%%%%%%%%%%%%%%%%%%%%%%%%
\section{Introduction}
\label{sec:intro}


\subsection{What is SUMMON}

SUMMON is a python extension module that provides rapid prototyping of 2D
visualizations.  By heavily relying on the python scripting language, SUMMON
allows the user to rapidly prototype a custom visualization for their data, 
without the overhead of a designing a graphical user interface or recompiling 
native code.  By simplifying the task of designing a visualization, users can 
spend more time on understanding their data. 

SUMMON was designed with several philosophies.  First, recompilation should
be avoided in order to speed up the development process.  Second, design of
graphical user interfaces should also be minimized.  Designing a good interface
takes planning and time to layout buttons, scrollbars, and dialog boxes.  Yet a 
poor interface is very painful to work with. Even when one has a good interface,
rarely can it be automated for batch mode.  Instead, SUMMON relies on the python
terminal for most interaction.  This allows the users direct access to  the
underlining code, which is more expressive, and can be automated through
scripting.  

Lastly, SUMMON is designed to be fast.  Libraries already exist for
accessing OpenGL in python.  However, python is relatively slow for real-time
interaction with large visualizations (trees with 100,000 leaves, matrices with
a million non-zeros, etc.).  Therefore, all real-time interaction is handled
with compiled native C++ code.  Python is only executed in the construction 
and occasional interaction with the visualization.  This arrangement provides 
the best of both worlds.




\subsection{Features}

Listed below is a short summary of the features offered in this version of
SUMMON.

\items{
    \item Python module extension
    \item Fast OpenGL graphics
    \item Drawing arbitrary points, lines, polygons, text with python scripting
    \item Binding inputs (keyboard, mouse, hotspots) to any python function 
    \item Separate threads for python and graphics (allows use of python prompt
          and responsive graphic at the same time)    
    \item Transparently handles graphics event loop, scrolling, zooming, text
          layout (auto-clipping, scaling, alignment), detecting clicks, allowing
          you to focus on viewing your data
    \item SVG output (also GIF/PNG/JPG/etc with ImageMagick)
    \item cross-platform (windows, linux)
}


%%%%%%%%%%%%%%%%%%%%%%%%%%%%%%%%%%%%%%%%%%%%%%%%%%%%%%%%%%%%%%%%%%%%%%%%%%%%
\section{Installing SUMMON}
\label{sec:installing}

The latest version of SUMMON can be obtained from 
http://people.csail.mit.edu/rasmus/summon/.  Download the *.tar.gz archive and
unzip it with the command:

\codeblock{tar zxvf summon-\version.tar.gz}

Before running or compiling SUMMON, the following libraries are required:
\items {
    \item python 2.4 (or greater)
    \item GL   
    \item GLUT
    \item SDL (for threading)
}

\subsection{Compiling SUMMON}

SUMMON can be installed using the standard distutils 
(http://docs.python.org/inst/inst.html).  For example, in the
\code{summon-\version} directory run:

\codeblock{python setup.py install}

To install SUMMON in another location other than in \code{/usr} use:

\codeblock{python setup.py install --prefix=<another directory prefix>}


\subsection{Configuring SUMMON}

SUMMON expects to find a configuration file called  \code{summon\_config.py}
somewhere in the python path.  Distutils installs a default module located in
your python path.  To customize SUMMON with your own key bindings and behavior,
you can write your own \code{summon\_config.py} file.  Just be sure it appears
in your python path somewhere {\em before} SUMMON default configuration file. 
Alternatively, you can create a configuration file \code{.summon\_config} in
your home directory.  The configuration file is nothing more than a python
script that calls the SUMMON function  \code{set\_binding} in order to
initialize the default keyboard and mouse  bindings.



%%%%%%%%%%%%%%%%%%%%%%%%%%%%%%%%%%%%%%%%%%%%%%%%%%%%%%%%%%%%%%%%%%%%%%%%%%%%
\section{Using SUMMON}
\label{sec:using}

SUMMON can be used as stand-alone program and as a module in a larger python
program.  The stand-alone version is installed in \code{PREFIX/bin/summon} and
is called from the command line as follows:

\codeblock{usage: summon [python script]}

On execution, SUMMON opens an OpenGL window and evaluates any script that it is
given in the python engine. After evaluation, the SUMMON prompt should appear
which provides direct access to the python engine.  Users should be familiar
with the python language in order to use SUMMON.

The SUMMON prompt acts exactly like the python prompt except for the OpenGL
window and the appearance of several automatically imported modules such as 
\code{summon}.  All of the commands needed to interact with the visualization
are within the \code{summon}  module.  

To learn how to use SUMMON, example scripts have been provided in the 
\code{summon/examples/} directory.  Examples of full fledged visualizations,
SUMMATRIX and SUMTREE, are also given in the \code{summon/bin/} directory. 
Their example input files are given in \code{summon/examples/summatrix/} and
\code{summon/examples/sumtree/}, respectively.



\subsection{Example Script}

For an introduction to the basic commands of SUMMON, let us walk through the
code of the first example.  To begin, change into the
\code{summon/examples/} directory and open up \code{01\_basics.py} in a
text editor.  Also use execute the example with following command.

\codeblock{\$ python 01\_basics.py}

The visualization should immediately appear in your OpenGL window.  The
following controls are available:

\vspace{.1in}
\begin{tabular}{ll}
    left mouse button          & scroll \\
    right mouse button         & zoom (down: zoom-out, up: zoom-in)\\
    Ctrl + right mouse button  & zoom x-axis \\
    Shift + right mouse button & zoom y-axis \\
    arrow keys                 & scroll \\
    Shift + arrow keys         & scroll faster \\
    Z                          & zoom in \\
    z                          & zoom out \\
    h                          & home (make all graphics visible) \\
    Ctrl + l                   & toggle anti-aliasing \\
    Crrl + p                   & output SVG of the current view \\
    Ctrl + Shift + p           & output PNG of the current view \\
    Ctrl + x                   & show/hide crosshair \\
    q                          & close window \\
\end{tabular}
\vspace{.2in}



In your text editor, the example \code{01\_basics.py} should contain the 
following python code:

\begin{minipage}{6in}
{ \footnotesize
\begin{verbatim}
#!/usr/bin/env python-i
# SUMMON examples
# 01_basics.py - basic commands

# make summon commands available
from summon.core import *
import summon

# syntax of used summon functions
# add_group( <group> )   = adds a group of graphics to the screen
# group( <elements> )    = creates a group from several graphical elements
# lines( x1, y1, x2, y2, ... )  = an element that draws one or more lines
# quads( x1, y1, ..., x4, y4, ... )  = an element that draws one or more quadrilaterals
# color( <red>, <green>, <blue>, [alpha] ) = a primitive that specifies a color


# create a new window
win = summon.Window("01_basics")

# add a line from (0,0) to (30,40)
win.add_group(lines(0,0, 30,40))

# add two blue quadrilaterals inside a group
win.add_group(group(color(0, 0, 1), 
                    quads(50,0, 50,70, 60,70, 60,0),
                    quads(65,0, 65,70, 75,70, 75,0)))

# add a multi-colored quad, where each vertex has it own color
win.add_group(quads(color(1,0,0), 100, 0,
                    color(0,1,0), 100, 70,
                    color(0,0,1), 140, 60,
                    color(1,1,1), 140, 0))


# add some text below everything else
win.add_group(text("Hello, world!",     # text to appear
                   0, -10, 140, -100,   # bounding box of text
                   "center", "top"))    # justification of text in bounding box

# center the "camera" so that all shapes are in view
win.home()

\end{verbatim}
}
\end{minipage}
\vspace{.25in}


The first line of the script imports the SUMMON module \code{summon} and all of 
the basic SUMMON functions (\code{group}, \code{lines}, \code{color}, etc) from
the \code{summon.core} module into the current environment.  
A new SUMMON graphics window is created using the \code{summon.Window} object.

All graphics are added and removed from the window in sets called {\em groups}. 
Groups provide a way to organize graphical elements into a hierarchy.
The first graphical group added to the window is a line.
The line is created with the \code{lines} function, which takes a series of
numbers specifying the end-point coordinates for the line.  The first
two numbers specify the x and y coordinates of one end-point (0,0) and the last
two specify the other end-point (30,40).  Next, the line is added to the window
using the \code{add\_group} function.

The next part of the example adds two quadrilaterals to the window with the
\code{quads} and \code{group} commands.  The arguments to the \code{quads}
function are similar to the \code{lines} function, except four vertices (8
numbers) are specified.  In the example, two quadrilaterals are created and
grouped together with the \code{group} function.

Note, both the \code{lines} and \code{quads} functions can draw multiple lines
and quadrilaterals (hence their plural names) by supplying more coordinates as
arguments.

The third group illustrates the use of color.  Color is stateful, as in OpenGL,
and all vertices that appear after a color object in a group will be affected. 
The \code{color} function creates a color object, which can appear
within graphical elements such as \code{lines} and \code{quads} or directly
inside a group.  Since each vertex in this example quad has a different color,
OpenGL will draw a quadrilateral that blends these colors.

Lastly, an example of text is shown.  Once again the text is added to the window
using the \code{add\_group} function.  The arguments to the text function
specify the text to be displayed, a bounding box specified by two
opposite  vertices, and then zero or more justifications ("left", "right",
"center", "top", "bottom", "middle") that will affect how the text aligns 
within its bounding box.  There are currently three types of text: \code{text}
(bitmap), \code{text\_scale} (stroke), \code{text\_clip} (stroked text that
clips).  The bitmap text will clip if it cannot fit within its bounding box. 
This is very useful in cases where the user zooms out very far and no more space
is available for the text to fit.  See the example \code{10\_text.py} for a
better illustration of the different text constructs.

The final function in the script is \code{win.home()}.  \code{win.home()} causes
the SUMMON window to scroll and zoom such that all graphics are visible.  This
is a very useful command for making sure that what you have drawn is visible in
the window.  The command can also be executed by pressing the 'h' key.  This key 
comes in handy when you "lose sight" of the visualization.

This is only a simple example.  See the remaining scripts for examples of
SUMMON's more powerful features.

\subsection{Example Visualizations: SUMMATRIX and SUMTREE}

In the \code{summon/bin/} directory are two programs, \code{summatrix} and
\code{sumtree} that use summon to visualize large datasets.  They are simply 
python scripts and so can be easily extended.  In my own work, I have 
extended the tree visualization program to integrate more closely with
biological data (executing CLUSTALW and MUSCLE on subtrees, displaying GO terms,
etc.).  The purpose of writing visualization programs in this way, is to allow
others to easily overlay and integrate their own data.  

Also in both visualizations the underling data is accessible through global
python variables.  That means if you have a very specific question like, "How
many genes in my subtree have a particular GO term?", you can quickly write a
few lines of python to walk the tree and answer the question yourself.  It would
be very difficult to anticipate all such questions during the development of a
non-scriptable visualization.

Example input files for both programs can be found under the 
\code{summon/examples} directory.  Both programs will print their usage if run
with no arguments.  View/execute the \code{view\_*.sh} scripts for examples of
how to call summatrix and sumtree.


\section{SUMMON Function Reference}

See \code{summon.html} for complete function reference.

\end{document}
