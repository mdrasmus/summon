\documentclass[12pt]{article}
\usepackage{amsmath}
\usepackage{graphicx}

% page dimensions
\textwidth=6.5in
\oddsidemargin=-0.0in
\evensidemargin=-0.0in
\topmargin=-0.2in
\footskip=0.8in
\textheight=8.00in


% list shortcuts
\newcommand{\enum}[1]{\begin{enumerate} #1 \end{enumerate}}
\newcommand{\items}[1]{\begin{itemize} #1 \end{itemize}}

% formatting macros
\newcommand{\bold}[1]{{\bf #1}}
\newcommand{\ul}[1]{\underline{#1}}
\newcommand{\code}[1]{{\tt #1}}
\newcommand{\codeblock}[1]{\vspace{.1in} {\tt #1} \vspace{.1in}}

% reference macros
\newcommand{\figref}[1]{Figure~\ref{#1}}
\newcommand{\secref}[1]{Section~\ref{#1}}
\newcommand{\algref}[1]{Algorithm~\ref{#1}}



\newcommand{\version}{1.0}


%%%%%%%%%%%%%%%%%%%%%%%%%%%%%%%%%%%%%%%%%%%%%%%%%%%%%%%%%%%%%%%%%%%%%%%%%%%%

\begin{document}

\begin{titlepage}

\begin{center}

\vspace*{2.5in}

{\huge \bf \fontfamily{phv}\selectfont 
SUMMON \version\ Manual
}
\vspace*{.5in}

{\large
Matt Rasmussen
}
\vspace*{.5in}

Computer Science and Artificial Intelligence Lab

Massachusetts Institute of Technology

\vspace*{.25in}

rasmus@mit.edu
\end{center}

\end{titlepage}


\tableofcontents

\clearpage

%%%%%%%%%%%%%%%%%%%%%%%%%%%%%%%%%%%%%%%%%%%%%%%%%%%%%%%%%%%%%%%%%%%%%%%%%%%%
\section{Introduction}
\label{sec:intro}


\subsection{What is SUMMON}

SUMMON is a general visualization program that uses an embedded python engine to
allow customization through scripting.  This allows the user to rapidly
prototype a custom visualization for their data, without the overhead of a
designing a graphical user interface or recompiling native code.  By simplifying
the task of designing a visualization, users can spend more time on
understanding their data. 

SUMMON was deisgned with several philosophies.  First, recompilation should
be avoided in order to speed up the development process.  Second, design of
graphical user interfaces should also be avoided.  Designing a good interface
takes planning and time to layout buttons, scrollbars, and dialog boxes.  Yet a 
poor interface is very painful to work with. Even when one has a good interface,
rarely can it be automated for batch mode.  Instead SUMMON relies on the
python terminal for most interaction.  This allows the users direct access to the
underlining code, which is more expressive, and can be automated through
scripting.  

Lastly, SUMMON is designed to be fast.  Libraries already exist for
accessing OpenGL in python.  However, python is relatively slow for real-time
interaction with large visualizations (trees with 100,000 leaves, matrices with
a million non-zeros, etc.).  Therefore, all real-time interaction is handled
with compiled native C++ code.  Python is only executed in the construction 
and occasional interaction with the visualization.  This arrangement provides 
the best of both worlds.




\subsection{Features}

Listed below is a short summary of the features offered in this version of
SUMMON.

\items{
    \item Embedded python engine
    \item Fast OpenGL graphics    
    \item Drawing arbitrary points, lines, polygons
    \item Binding inputs (keyboard, mouse, hotspots) to any python function 
    \item SVG output (also PNG with ImageMagick)
    \item cross-platform (windows, linux)
}

%%%%%%%%%%%%%%%%%%%%%%%%%%%%%%%%%%%%%%%%%%%%%%%%%%%%%%%%%%%%%%%%%%%%%%%%%%%%
\section{Installing SUMMON}
\label{sec:installing}

Before running or compiling SUMMON, the following libraries are required:
\items {
    \item python 2.3 (or greater)
    \item GL/GLU    
    \item glut 
    \item SDL (for threading)
    \item pthread (for linux only)
    \item readline/ncurses
    \item cygwin (for windows only)
}

Precompiled versions of SUMMON are available in \code{summon/bin/} for linux
(\code{summon}) and windows (\code{summon.exe}).  To install SUMMON either add
\code{summon/bin/} to your \code{PATH} environment variable or copy all the
programs in \code{summon/bin/} to a directory on your \code{PATH}.  Windows users
will need to run SUMMON in the CYGWIN linux environment for windows 
(http://www.cygwin.com/).

\subsection{Compiling SUMMON}

To compile your own version of SUMMON, use the Makefile available under 
\code{summon/}.  Below several commands are given that should compile SUMMON for
your system.

To compile the linux version change into the \code{summon/} directory and
execute the following command.

\codeblock{\$ make}

To copy the compiled linux version of SUMMON to \code{summon/bin/} type

\codeblock{\$ make install}

To compile the windows version of SUMMON and copy it to \code{summon/bin/} type

\codeblock{\$ make summon.exe}



\subsection{Configuring SUMMON}

SUMMON requires access to its own python package in order to operate properly.
In addition, SUMMON expects to find a configuration file called 
\code{summon\_config.py} somewhere in the python path.  The SUMMON package, 
\code{summonlib}, and a default configuration file appear under 
\code{summon/lib/}.  Therefore, before attempting to run SUMMON add 
\code{summon/lib/} to your \code{PYTHONPATH} environment variable.  For example,
if you use bash as a shell then add the following to your \code{.bashrc} file:

\codeblock{
    export PYTHONPATH=\$PYTHONPATH:location\_of\_your\_choice/summon/lib
}

To customize SUMMON with your own key bindings and behavior, you can write your
own \code{summon\_config.py} file.  Just be sure it appears in your python path
somewhere {\em before} SUMMON default configuration file.  The configuration file
is nothing more than a python script that calls the SUMMON function 
\code{set\_binding} in order to initialize the default keyboard and mouse 
bindings.



%%%%%%%%%%%%%%%%%%%%%%%%%%%%%%%%%%%%%%%%%%%%%%%%%%%%%%%%%%%%%%%%%%%%%%%%%%%%
\section{Using SUMMON}
\label{sec:using}

SUMMON should be run from the commandline.  I recommend Windows users to use
CYGWIN for running SUMMON from the commandline.  SUMMON itself takes a single
optional python script as an argument.  On execution, SUMMON opens an OpenGL
window and evaluates any script that it is given in the python engine. After
evaluation, the SUMMON prompt should appear which provides direct access to the
python engine.  Users should be familar with the python language in order to use
SUMMON.

\codeblock{usage: summon [python script]}


The SUMMON prompt acts exactly like the python prompt except for the OpenGL
window and the appearence of a builtin module called \code{summon}.  All of the
commands needed to interact with the visualization are within the \code{summon} 
module.  

To learn how to use SUMMON, example scripts have been provided in the 
\code{summon/examples/summon/} directory.  Examples of full fledged
visualizations, SUMMATRIX and SUMTREE, are also given in the \code{summon/bin/}
directory.  Their example input files are given in
\code{summon/examples/summatrix/} and \code{summon/examples/sumtree/},
respectively.



\subsection{Example Script}

For an introduction to the basic commands of SUMMON, let us walk through the
code of the first example.  To begin, change into the
\code{summon/examples/summon/} directory and open up \code{example1.py} in a
text editor.  Also use SUMMON to execute the example with following command.

\codeblock{\$ summon example1.py}

The visualization should immediately appear in your OpenGL window.  The
following controls are available:

\vspace{.1in}
\begin{tabular}{ll}
    left mouse button          & scroll \\
    right mouse button         & zoom \\
    Ctrl + right mouse button  & zoom x-axis \\
    Shift + right mouse button & zoom y-axis \\
    arrow keys                 & scroll \\
    Shift + arrow keys         & scroll faster \\
    Z                          & zoom in \\
    z                          & zoom out \\
    h                          & home (make all graphics visible) \\
    l                          & toggle anti-aliasing \\
    p                          & output SVG of the current view \\
    Ctrl + p                   & output PNG of the current view \\
    q                          & quit \\
\end{tabular}
\vspace{.2in}



In your text editor, the example \code{example1.py} should contain the 
following python code:

\begin{minipage}{6in}
{ \footnotesize
\begin{verbatim}
# SUMMON examples
# example1.py - basic commands

# make summon commands available
from summon import *

# syntax of used summon functions
# add_group( <group> )   : adds a group of graphics to the screen
# group( <elements> )    : creates a group from several graphical elements
# lines( <primitives> )  : an element that draws one or more lines
# quads( <primitives> )  : an element that draws one or more quadrilaterals
# color( <red>, <green>, <blue>, [alpha] ) : a primitive that specifies a color


# clear the screen of all drawing
clear_groups()

# add a line from (0,0) to (30,40)
add_group(group(lines(0,0, 30,40)))

# add a quadrilateral
add_group(group(quads(50,0, 50,70, 60,70, 60,0)))

# add a multi-colored quad
add_group(group(quads(
    color(1,0,0), 100, 0,
    color(0,1,0), 100, 70,
    color(0,0,1), 140, 60,
    color(1,1,1), 140, 0)))


# add some text below everything else
add_group(group(
    text("Hello, world!",    # text to appear
         0, -10, 140, -100,    # bounding box of text
         "center", "top")))    # justification of text in bounding box

# center the "camera" so that all shapes are in view
home()
\end{verbatim}
}
\end{minipage}
\vspace{.25in}


As you can see, the first line of the script imports all of the SUMMON functions
from the \code{summon} module into the current environment.  The first of such
functions is the \code{clear\_groups()} command.  All graphics are added and
removed from the screen in sets called {\em groups}.  Groups provide a way to
organize and reference graphical elements.  The \code{clear\_groups()} function
removes all groups that may be on the screen.

The next line of python code in the example adds a single line to the screen. 
The line is created with the \code{lines} function, which takes a series of
numbers specifying the end-point coordinates for the line.  The first
two numbers specify the x and y coordinates of one end-point (0,0) and the last
two specify the other end-point (30,40).  Next, the line is placed in a group
using the \code{group} function which returns a group ready to be added to the
screen.   Lastly, the \code{add\_group} function is called on the group.  This
function finally places the line on the screen.  Although this may seem like a
lot to type to draw a single line, in most uses several lines and other graphics 
are placed a group before adding them to the screen.

The next line in the example adds a quadrilateral to the screen with the
\code{quads} command.  The arguments to the \code{quads} function are similar to
the \code{lines} function, except four vertices (8 numbers) are specified.  Both
functions can draw multiple lines and quadrilaterals (hence their plural names)
by supplying more coordinates as arguments.

The third group illustrates the use of color.  Color is stateful, as in OpenGL,
and all vertices that appear after a color object in a group will be affected. 
The \code{color} function creates a color object.  Color objects can appear
within graphical elements such as \code{lines} and \code{quads} or directly
inside a group.  Since each vertex in this example quad has a different color,
OpenGL will draw a quadrilateral that blends these colors.

Lastly, an example of text is shown.  Once again the text is added to the screen
using the \code{add\_group} function.  The arguments to the text function
specify the text to be displayed, a bounding box specified by two
opposite  vertices, and then zero or more justifications ("center", "top",
etc.) that will affect how the text aligns in its bounding box.  There are
currently three types of text: \code{text} (bitmap), \code{text\_scale} (stroke),
\code{text\_clip} (stroked text that clips).  The bitmap text will clip if it
cannot fit within its bounding box.  This is very useful in cases where the user
zooms out very far and no more space is available for the text to fit.  See the
example \code{text.py} for a better illustration of the different text
constructs.

The final function in the script is \code{home()}.  \code{home()} causes the
SUMMON window to scroll and zoom such that all graphics are visible.  This is a
very useful command for making sure what you have drawn is visible in the
window.  The command can also be execute by pressing the 'h' key.  This key 
comes in handy when you "lose sight" of the visualization.

This is only a simple example.  See the remaining scripts for examples of
SUMMON's more powerful features.

\subsection{Example Visualizations: SUMMATRIX and SUMTREE}

In the \code{summon/bin/} directory are two programs, \code{summatrix} and
\code{sumtree} that use summon to visualize large datasets.  There programs are
written in python and so can be easily extended.  In my own work, I have 
extended the tree visualization program to integrate more closely with biological
data (executing CLUSTALW and MUSCLE on subtrees, displaying GO terms, etc.).  The
purpose of writing visualization programs in this way, is to allow others to
easily overlay and integrate their own data.  

Also in both visualizations the underling data is accessable through global python
variables.  That means if you have a very specific question like, "How many
genes in my subtree have a particular GO term?", you can quickly write a few
lines of python to walk the tree and answer the question youself.  It would be
very difficult to anticipate all such questions during the development of a
visualization.  And yet when visualizing, it can become frustrating if you cannot
fully interact with the  data.

Example input files for both programs can be found under the 
\code{summon/examples} directory.  Both programs will print their usage if run
with no arguments.  Here are some recommended examples:

\codeblock{\$ sumtree -n olfactory-genes.tree}

\codeblock{\$ sumtree -n olfactory-genes.tree -t 10}

\codeblock{\$ summatrix -i human\_mouse.imat}


\section{SUMMON Function Reference}

All help information is also available from the SUMMON prompt.  Use 
\code{help(command)} to get required arguments and a usage description.

%%%%%%%%%%%%%%%%%%%%%%%%%%%%%%%%%%%%%%%%%%%%%%%%%%%%%%%%%%%%%%%%%%%%%%
\subsection{SUMMON General Functions}

{\tt add\_group(group) }

adds drawing groups to the current model \\


{\tt assign\_model(windowid, 'world'|'screen', modelid) }

assigns a model to a window \\


{\tt call\_proc(proc) }

executes a procedure that takes no arguments \\


{\tt clear\_all\_bindings() }

clear all bindings for all input \\


{\tt clear\_binding(input) }

clear all bindings for an input \\


{\tt clear\_groups() }

removes all drawing groups from the current display \\


{\tt close\_window([id]) }

closes a window \\


{\tt del\_model(modelid) }

deletes a model \\


{\tt focus(x, y) }

focus the view on (x,y) \\


{\tt get\_bgcolor() }

gets background color \\


{\tt get\_group(groupid) }

creates a tuple object that represents a group \\


{\tt get\_model(windowid, ['world'|'screen']) }

gets the model id of a window \\


{\tt get\_models() }

gets a list of ids for all models \\


{\tt get\_mouse\_pos('world'|'screen'|'window') }

gets the current mouse position in the requested coordinates \\


{\tt get\_root\_id() }

gets the group id of the root group \\


{\tt get\_visible() }

gets visible bounding box \\


{\tt get\_window() }

gets the id of the current window \\


{\tt get\_window\_name(id) }

get the name of a window \\


{\tt get\_window\_size() }

gets current window's size \\


{\tt get\_windows() }

gets a list of ids for all open windows \\


{\tt home() }

adjust view to show all graphics \\


{\tt insert\_group(groupid, group) }

inserts drawing groups under an existing group \\


{\tt new\_model() }

creates a new model and returns its id \\


{\tt new\_window() }

creates a new window and returns its id \\


{\tt redraw\_call(func) }

calls function 'func' on every redraw \\


{\tt remove\_group(groups) }

removes drawing groups from the current display \\


{\tt replace\_group(groupid, group) }

replaces a drawing group on the current display \\


{\tt set\_antialias(True|False) }

sets anti-aliasing status \\


{\tt set\_bgcolor(red, green, blue) }

sets background color \\


{\tt set\_binding(input, proc|command\_name) }

bind an input to a command or procedure \\


{\tt set\_model(modelid) }

sets the current model \\


{\tt set\_visible(x1, y1, x2, y2) }

change display to contain region (x1,y1)-(x2,y2) \\


{\tt set\_window(id) }

sets the current window \\


{\tt set\_window\_name(id name) }

sets the name of a window \\


{\tt set\_window\_size(x, y) }

sets current window's size \\


{\tt show\_group(groupid, True|False) }

sets the visibilty of a group \\


{\tt timer\_call(delay, func) }

calls a function 'func' after a delay in seconds \\


{\tt trans(x, y) }

translate the view by (x,y) \\


{\tt version() }

prints the current version \\


{\tt vertices(x, y, * more) }

creates a list of vertices \\


{\tt zoom(factorX, factorY) }

zoom view by a factor \\


{\tt zoomx(factor) }

zoom x-axis by a factor \\


{\tt zoomy(factor) }

zoom y-axis by a factor \\


%%%%%%%%%%%%%%%%%%%%%%%%%%%%%%%%%%%%%%%%%%%%%%%%%%%%%%%%%%%%%%%%%%%%%%
\subsection{SUMMON Graphics}

{\tt points(* vertices|colors) }

plots vertices as points \\


{\tt lines(* vertices|colors) }

plots vertices as lines \\


{\tt line\_strip(* vertices|colors) }

plots vertices as connected lines \\


{\tt triangles(* vertices|colors) }

plots vertices as triangles \\


{\tt triangle\_strip(* vertices|colors) }

plots vertices as connected triangles \\


{\tt triangle\_fan(* vertices|colors) }

plots vertices as triangles in a fan \\


{\tt quads(* vertices|colors) }

plots vertices as quads \\


{\tt quad\_strip(* vertices|colors) }

plots vertices as connected quads \\


{\tt polygon(* vertices|colors) }

plots vertices as a convex polygon \\


%%%%%%%%%%%%%%%%%%%%%%%%%%%%%%%%%%%%%%%%%%%%%%%%%%%%%%%%%%%%%%%%%%%%%%
\subsection{SUMMON Primitives}

{\tt vertices(x, y, * more) }

creates a list of vertices \\


{\tt color(red, green, blue, [alpha]) }

creates a color from 3 or 4 values in [0,1] \\


%%%%%%%%%%%%%%%%%%%%%%%%%%%%%%%%%%%%%%%%%%%%%%%%%%%%%%%%%%%%%%%%%%%%%%
\subsection{SUMMON Text constructs}

{\tt text(string, x1, y1, x2, y2, ['left'|'center'|'right'], ['top'|'middle'|'bottom']) }

draws text justified within a bounding box \\


{\tt text\_scale(string, x1, y1, x2, y2, ['left'|'center'|'right'], ['top'|'middle'|'bottom']) }

draws stroked text within a bounding box \\


{\tt text\_clip(string, x1, y1, x2, y2, minheight, maxheight, ['left'|'center'|'right'], ['top'|'middle'|'bottom']) }

draws stroked text within a bounding box and height restrictions \\


%%%%%%%%%%%%%%%%%%%%%%%%%%%%%%%%%%%%%%%%%%%%%%%%%%%%%%%%%%%%%%%%%%%%%%
\subsection{SUMMON Transforms}

{\tt translate(x, y, * elements) }

translates the coordinate system of enclosed elements \\


{\tt rotate(angle, * elements) }

rotates the coordinate system of enclosed elements \\


{\tt flip(x, y, * elements) }

flips the coordinate system of enclosed elements over (x,y) \\


{\tt scale(x, y, * elements) }

scales the coordinate system of enclosed elements \\


%%%%%%%%%%%%%%%%%%%%%%%%%%%%%%%%%%%%%%%%%%%%%%%%%%%%%%%%%%%%%%%%%%%%%%
\subsection{SUMMON Input specifications}

{\tt input\_motion('left'|'middle'|'right', 'up'|'down', ['shift'], ['ctrl'], ['alt']) }

specifies a mouse motion input \\


{\tt input\_key(key, ['shift'], ['ctrl'], ['alt']) }

specifies a keyboard input \\


{\tt input\_click('left'|'middle'|'right', 'up'|'down', ['shift'], ['ctrl'], ['alt']) }

specifies a mouse click input \\


%%%%%%%%%%%%%%%%%%%%%%%%%%%%%%%%%%%%%%%%%%%%%%%%%%%%%%%%%%%%%%%%%%%%%%
\subsection{SUMMON Miscellaneous}

{\tt hotspot('over'|'out'|'click', x1, y1, x2, y2, proc) }

constructs hotspot for a region that activates a python procedure 'proc' \\


{\tt hotspot\_click(cannot be invoked on commandline) }

activates a hotspot with a 'click' action \\





\end{document}
